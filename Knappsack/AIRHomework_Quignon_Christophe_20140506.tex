%\documentclass{beamer}
%\usetheme{Pittsburgh} 
\documentclass{scrartcl}

\usepackage[utf8]{inputenc}
\usepackage{default}
\usepackage[procnames]{listings}
\usepackage{graphicx}
%\usepackage[toc,page]{appendix}
\usepackage{caption}
\usepackage{hyperref}
\usepackage{color}

%Python
\definecolor{keywords}{RGB}{255,0,90}
\definecolor{comments}{RGB}{0,0,113}
\definecolor{red}{RGB}{160,0,0}
\definecolor{green}{RGB}{0,150,0}
\lstset{language=Python, 
    basicstyle=\ttfamily\scriptsize, 
    keywordstyle=\color{keywords},
    commentstyle=\color{comments},
    stringstyle=\color{red},
    identifierstyle=\color{green},
    procnamekeys={def,class},
    breaklines=true,
    columns=fullflexible,
    %Numbering and Tabs
    tabsize=4,
    showspaces=false,
    showstringspaces=false}


\begin{document}

\title{Artificial Intelligence for Robots}
\subtitle{Homework 4}
\author{
  Quignon, Christophe \\
  %Familyname, Name
} 
%\institute{Hochschule Bonn Rhein Sieg}
\date{\today}

\maketitle{}
\section{Task}Your task is to implement a solver of the knapsack problem on the provided data, by using a Local
Search Algorithm, such as Hill Climbing or Simulated Annealing. Your program should output which
objects will put in the knapsack, what is the total weight of the objects, and what is the total profit.
\\
Hillclimbing was choosen.
\newpage
\section{Code}
\lstinputlisting[language=Python]{knappsack.py}

\newpage
\section{Output}
Please note that the algorithm is based on random decisions. Therefore the output will vary.
\lstinputlisting[language=C++]{knappsack.out}

\end{document}
