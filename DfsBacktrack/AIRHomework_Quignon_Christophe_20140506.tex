%\documentclass{beamer}
%\usetheme{Pittsburgh} 
\documentclass{scrartcl}

\usepackage[utf8]{inputenc}
\usepackage{default}
\usepackage[procnames]{listings}
\usepackage{graphicx}
%\usepackage[toc,page]{appendix}
\usepackage{caption}
\usepackage{hyperref}
\usepackage{color}

%Python
\definecolor{keywords}{RGB}{255,0,90}
\definecolor{comments}{RGB}{0,0,113}
\definecolor{red}{RGB}{160,0,0}
\definecolor{green}{RGB}{0,150,0}
\lstset{language=Python, 
    basicstyle=\ttfamily\scriptsize, 
    keywordstyle=\color{keywords},
    commentstyle=\color{comments},
    stringstyle=\color{red},
    identifierstyle=\color{green},
    procnamekeys={def,class},
    breaklines=true,
    columns=fullflexible,
    %Numbering and Tabs
    tabsize=4,
    showspaces=false,
    showstringspaces=false}


\begin{document}

\title{Artificial Intelligence for Robots}
\subtitle{Homework 4}
\author{
  Quignon, Christophe \\
  %Familyname, Name
} 
%\institute{Hochschule Bonn Rhein Sieg}
\date{\today}

\maketitle{}
\section{Tasks}
\begin{enumerate}
  \item Implement an appropriate solver for the given constraint satisfaction problem, e.g. depth-first-search with backtracking.
  \item Implement the following three ordering strategies for the successor functions:
  \begin{itemize}
    \item Line number in the list
    \item Euclidean distance
    \item Time to deadline
  \end{itemize}
  \item Evaluate the performance of the algorithm and the different ordering strategies, e.g. like numbers
of evaluations and path length for each of the given scenarios. An evaluation is performed any
time the constraints associated with a partial path are checked for satisfaction.
\end{enumerate}
\newpage
\section{Evaluation}

The sorting has a big impact on the performance. But no given sorting is always better than the others.


\subsection{File: scenarios/scenario1.txt}\begin{tabular}{ l | l l} \\Sorting &Time & Evaluations\\\hline
\\ Line number &
0.00534582138062 ms
& 157
\\ Distance &
0.00103807449341 ms
& 24
\\ Deadline &
0.000430822372437 ms
& 6

\end{tabular} \subsection{File: scenarios/scenario2.txt}\begin{tabular}{ l | l l} \\Sorting &Time & Evaluations\\\hline
\\ Line number &
0.00589418411255 ms
& 82
\\ Distance &
0.0968480110168 ms
& 2358
\\ Deadline &
0.0720450878143 ms
& 1962

\end{tabular} \subsection{File: scenarios/scenario3.txt}\begin{tabular}{ l | l l} \\Sorting &Time & Evaluations\\\hline
\\ Line number &
0.00799703598022 ms
& 265
\\ Distance &
0.0237109661102 ms
& 816
\\ Deadline &
0.00121307373047 ms
& 23

\end{tabular} \subsection{File: scenarios/scenario4.txt}\begin{tabular}{ l | l l} \\Sorting &Time & Evaluations\\\hline
\\ Line number &
8.5898399353 ms
& 325655
\\ Distance &
8.71894693375 ms
& 325655
\\ Deadline &
8.47428703308 ms
& 325655

\end{tabular} \subsection{File: scenarios/scenario5.txt}\begin{tabular}{ l | l l} \\Sorting &Time & Evaluations\\\hline
\\ Line number &
0.00313687324524 ms
& 25
\\ Distance &
0.00267004966736 ms
& 15
\\ Deadline &
0.320579051971 ms
& 5518
\end{tabular} \\

\newpage
\section{Code}
dfsbacktracking.py
\lstinputlisting[language=Python]{dfsbacktracking.py}

\newpage
\section{Output}
File 4 has no solution and loops forever. Therefore it is not in the output.
The sorting order is:
  \begin{itemize}
    \item Line number in the list
    \item Euclidean distance
    \item Time to deadline
  \end{itemize}
\lstinputlisting{out.out}

\end{document}
