%\documentclass{beamer}
%\usetheme{Pittsburgh} 
\documentclass{scrartcl}

\usepackage[utf8]{inputenc}
\usepackage{default}
\usepackage[procnames]{listings}
\usepackage{graphicx}
%\usepackage[toc,page]{appendix}
\usepackage{caption}
\usepackage{hyperref}
\usepackage{color}
\usepackage{enumitem} 

%Python
\definecolor{keywords}{RGB}{255,0,90}
\definecolor{comments}{RGB}{0,0,113}
\definecolor{red}{RGB}{160,0,0}
\definecolor{green}{RGB}{0,150,0}
\lstset{language=Python, 
    basicstyle=\ttfamily\scriptsize, 
    keywordstyle=\color{keywords},
    commentstyle=\color{comments},
    stringstyle=\color{red},
    identifierstyle=\color{green},
    procnamekeys={def,class},
    breaklines=true,
    columns=fullflexible,
}


\begin{document}

\title{Artificial Intelligence for Robots}
\subtitle{Homework 9}
\author{
  Quignon, Christophe \\
  %Familyname, Name
} 
%\institute{Hochschule Bonn Rhein Sieg}
\date{\today}

\maketitle{}
%\section{Tasks}
\section{AIMA 9.9 Write down logical representations of the following sentences, suitable for use with Generalized Modus Ponens: }

\begin{enumerate}[label={\alph*)}] 
	\item Horses, cows and pigs are mammals.\\
	$\forall x \quad Horse(x) \Rightarrow Mammal(x) $\\
	$\forall x \quad Pig(x) \Rightarrow Mammal(x) $\\
	$\forall x \quad Cow(x) \Rightarrow Mammal(x) $
	\item An offspring of a horse is a horse.\\
	$\forall h, o \quad Horse(h) \land Offspring(h, o) \Rightarrow Horse(o) $
	\item Bluebeard is a horse.\\
	$Horse(Bluebeard)$
	\item Bluebeard is Charlie’s parent.\\
	$Parent(Bluebeard, Charlie)$
	\item Offspring and parent are inverse relations.\\
	$\forall x, y \quad Offspring(x, y) \iff Parent(Y, x)$
	\item Every mammal has a parent.\\
	$\forall m \exists p \quad Mammal(m) \land Parent(m, p)$
\end{enumerate}

\section{ AIMA 9.10 In this question we will use the sentences you wrote in Exercise 9.9 to answer a question using a backward-chaining algorithm.}

\begin{enumerate}[label={\alph*)}] 
	\item Draw the tree generated by using backward chaining for the query $\exists h \quad horse(h)$\\
	$Horse(h)$\\
	$Offspring(h, o)\quad\quad\quad Horse(Bluebeard)$\\
	$Parent(h, o)\quad\quad\quad Offspring(Bluebeard, p) \quad \quad Horse(p)$\\
	$y\gets Bluebeard$\quad\quad$Parent(p, Bluebeard)\quad \quad \dots$\\
	$o\gets Charlie$\quad\quad$\dots$\\
	
	\item What do you notice about this domain?\\
	The Offspring Parent relation keep repeating infinite.

	\item How many solutions for h actually follow from your sentences?\\
	Inifitely many.

	\item Can you think of a way to find them all? (Hint: you might want to consult Smith et al (1986)).\\
	Initialise an ancestor relationship.


\end{enumerate}


\section{ AIMA 9.18 From “Horses are animals”, it follows that “The head of a horse is the head of an animal”. Demonstrate that this inference is valid by carrying out the following steps: }

\begin{enumerate}[label={\alph*)}] 
	\item Translate the premise and the conclusion into the language of first-order logic. Use three predicates: $HeadOf(h, x)$ (Meaning “h is head of x ”), $Horse(x)$ and $Animal(x)$.\\
	$\forall x\quad Horse(x) \Rightarrow Animal(x) $\\
	$\forall x h\quad [(Horse(x)\land HeadOf(h, x)))\Rightarrow \exists a \quad (Animal(a)\land HeadOf(h, a))]$
	
	\item Negate the conclusion, and convert the premise and the negated conclusion into conjunctive normal form.\\
	\begin{enumerate}
		\item $\neg Horse(x) \lor Animal(x)$
		\item $Horse(C1)$
		\item $HeadOf(C2, C1)$
		\item $\neg Animal(y)\land \neg HeadOf(C2, y)$
 	\end{enumerate}
	
	\item Use resolution to show that the conclusion follows from the premise.\\
	From c and d: $\neg Animal(C1)$\\
	With a $\neg Horse(C1)$\\
	With b $\bot$
\end{enumerate}


\section{Time estimate}
\begin{itemize}
	\item AIMA 9.09\\
	1h
	\item AIMA 9.10\\
	1h
	\item AIMA 9.18\\
	 1h
\end{itemize}

	

\end{document}
